\section*{Zusammenfassung}

\addcontentsline{toc}{section}{Zusammenfassung}


Thema dieser Arbeit ist die Entwicklung einer robusten Kommunikation zwischen zwei \ac{uC} zur Weiterleitung von sensiblen Messdaten und 
die anschließende Weiterleitung der Daten an ein Netzwerk. Die Entwicklung soll an Hand eines Praxisbeispiels demonstriert werden.

\smallskip

Grundlage der Arbeit ist eine vorhergegangene Studienarbeit \citep{IoTGateway} in welcher die Hardware,
welche die Grundlage für dieses Projekt bildet, entwickelt, bestückt und in Betrieb genommen wurde.

Die entwickelte Platine verfügt über zwei \ac{uC}, deren \acs{UART}-Schnittstellen miteinander verbunden sind. Ein Prozessor ist hierbei
für die Verarbeitung von Sensormesswerten zuständig, während der andere als \acs{WLAN}-Relais an ein übergeordnetes Netzwerk fungiert.

\smallskip

Auf Basis der \acs{UART}-Schnittstellen der beiden Prozessoren soll ein Protokoll implementiert werden, welches einerseits den Transport
von Messwerten in die eine Richtung, als auch den Transport von Konfigurationsbefehlen in die andere Richtung ermöglicht. 

Als Demonstrationsbeispiel wurde die kapazitive Messung von Feuchtigkeit im Erdreich gewählt. Die gemessenen Daten sollen verarbeitet werden und
per \ac{MQTT} an ein übergeordnetes Netzwerk weitergeleitet werden.
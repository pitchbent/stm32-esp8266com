\subsection{Protokoll}

Im folgenden soll die Entwicklung des Protokolls erklärt werden. Zuerst wird auf den Algorithmus zur zyklischen Redundanzprüfung eingegangen,
anschließend auf den Aufbau der übertragenen Nachrichten.

\subsubsection{CRC16-CCITT}

Als Implementation der zyklische Redundanzprüfung wurde die Version \textit{CRC16-CCITT} gewählt, da diese bewährt und gut dokumentiert ist. Leider 
kursieren viele fehlerhafte Implementationen dieses Algorithmus - es wurde jedoch wert darauf gelegt, die Richtige Version zu implementieren.

\smallskip

Das CRC-Polynom lautet $x^{16}+x^{15}+x^2+1$. Die hexadezimale Repräsentation ergibt sich deshalb zu $0x1021$.
\smallskip

Als Startwert für die Berechnung wird $0x0$ gewählt - fehlerhafte Implementationen beginnen oftmal mit $0xFFFF$. 

\begin{wraptable}{l}{30mm} 
    \centering
\begin{tabular}{ |c | c | c | }
    \hline
    a & b & y \\
    \hline \hline
    0 & 0 & 0 \\
    \hline
    0 & 1 & 1 \\
    \hline
    1 & 0 & 1 \\
    \hline
    1 & 1 & 0 \\
    \hline
\end{tabular}
\caption{XOR}
\end{wraptable}

\smallskip

Zwar handelt es sich bei der Berechnung des CRC-Wertes um eine Division, allerdings um eine \textit{Polynom}divsion - diese wird mit Hilfe eines Exklusiv-Oder
Gatters (\textit{XOR}) und Schieberegistern realisiert. Den zu prüfenden Daten werden abhängig von der Länge des CRC-Polynoms Nullen angehängt, in diesem Falle also 
16 Stück.

\smallskip

Um die Daten zu Prüfen, wird das CRC-Schieberegister mit '0' initialisiert. Anschließend wird der zu prüfende Wert, mit angehängten Nullen von rechts
in das Schieberegister "geschoben", bis das \ac{MSB} gleich '1' ist. Anschließend wird das Schieberegister um eine Einheit weitergeschoben, sodass das \ac{MSB}
herausfällt. Dann wird das Schieberegister mit Hilfe des XOR-Vergleiches mit dem CRC-Polynom verglichen. Der so entstandene Wert wird wieder in das Schieberegister
übernommen \citep{IK_VL}. 

\smallskip

Nun werden immer wieder Daten von rechts in das Register hineingeschoben. Immer wenn das \ac{MSB} einer '1' entspricht und im nächsten Schritt aus dem Register
geschoben wird, wird ein XOR-Vergleich mit dem CRC-Polynom durchgeführt. Dies wird so lange wiederholt, bis keine neuen Daten mehr in das Register geschoben werden können.
Der Wert welcher nun im Schieberegister verbleibt, entspricht der Prüfsumme \citep{IK_VL}.

\smallskip

Die Implementation in C bedient sich zweier Kniffe, um die zuvor erklärte Berechnung zu beschleunigen. Auf ein Initialisieren mit 0x0000 kann verzichtet werden,
da keine XOR-Vergleiche mit Nullen durchgeführt werden. Der Startwert wird also direkt mit den Eingangsdaten initialisiert werden. Auch auf ein anhängen der Nullen 
kann verzichtet werden, da der \textit{<<-Operator (left shift)} automatisch am \ac{LSB} Nullen anhängt. 

\newpage

Umgesetzt in C Code entsteht folgende Funktion: 

\begin{lstlisting}[caption={\textit{Berechnung CRC16}}]
uint16_t CRC16_buf(const uint8_t * pBuf, uint16_t len) 
{
const uint16_t poly = 0x1021;
uint16_t crc = 0;

for (uint8_t i = 0; i < len; i++)
{
    crc ^= pBuf[i] << 8; //Move Byte into 16Bit CRC Register

    for(uint8_t j = 0; j < 8; j++)
    {
        if((crc & 0x8000) != 0) //Test for MSB
        {
            crc = (crc<<1) ^ poly; //If MSB = 1 shift & XOR
        }
        else
            crc <<= 1;	//If not just shift
    }
}
return crc;
}
\end{lstlisting}

Der Funktion wird ein Zeiger zu einem Array übergeben, sowie die Länge dieses Arrays. Die Variable \lstinline!poly! repräsentiert das CRC-Polynom,
während die Variable \lstinline!crc! für das Schieberegister steht.

\smallskip

Die Berechnung des CRC-Wertes wird für jedes Byte des Arrays durchgeführt, wobei der Startwert für jedes Byte nach dem ersten die CRC-Prüfsumme des 
letzten Durchlaufes ist. Für jedes Byte wiederrum müssen, auf Grund der länge eines Bytes, acht mal die Shift- und XOR-Operationen durchgeführt werden.

\smallskip

Mittels des Wertes \lstinline!0x8000! wird geprüft, ob das \ac{MSB} gesetzt ist. Der Rückgabewert entspricht der CRC-Prüfsumme. 

\newpage

\subsubsection{Datenformat}
\label{subsub: Datenformat}

Bei den übertragenen Daten handelt es sich um Zeichen im \acs{ASCII}-Format. Dies vereinfacht die Auswertung und hat den Vorteil, dass die Kommunikation
zu Testzwecken leicht mitgelesen werden kann. Im folgenden wird unter \textit{Telegramm} die Gesamtheit der übertragenen Daten verstanden, während 
der Ausdruck \textit{Nachricht} den eigentlichen Informationsgehalt beschreibt.

\smallskip

Ein Telegramm teilt sich in mehrere Teile auf:

\begin{itemize}
    \item Startzeichen
    \item Länge
    \item Nachricht
    \item CRC-Prüfsumme
\end{itemize}



Die verschiedenen Teile bestehen aus einer verschiedenen Anzahl an Bytes. Während Start- und Endzeichen nur ein Byte benötigen, werden für die
Länge der Nachricht und die CRC-Prüfsumme zwei Bytes benötigt.

\smallskip

\begin{wraptable}{r}{75mm} 
    
    \begin{tabular}{ |c | c | c | c | c | c | c | }
        \hline
        1 & 2 & 3 & 4 & n+4 & n+5 & n+6  \\
        \hline \hline
        Start  & \multicolumn{2}{|c|}{Länge} & \multicolumn{2}{|c|}{Nachricht} & \multicolumn{2}{|c|}{CRC16} \\
        \hline
    
      \end{tabular}
    \centering
    \caption{Telegramm}
    \end{wraptable}


Da die Länge der Nachricht aus zwei Bytes besteht, ergibt sich eine maximale Nachrichtenlänge von 99 Bytes. Auf ein Endzeichen wird verzichtet - 
das Ende des Telegramms berechnet sich aus der Länge der Nachricht. Das Startzeichen entspricht dem \acs{ASCII}-Zeichen '<'.


\subsection{Fazit}
Es wurde eine robuste Übertragung sensibler Messwerte erfolgreich implementiert. Mittels der genutzten Technologien wie zyklischer Redundanzprüfung, \ac{UART} und \ac{MQTT}
werden die Messwerte zuverlässig übertragen und auf Fehler in der Übertragung überprüft. Durch die Nutzung von \ac{MQTT} ist die Anbindung an ein übergeordnetes 
Netzwerk möglich. 

\smallskip

Durch die Implementation eines kapazitiven Feuchtigkeitssensors wurde die Funktionalität der Übertragung erfolgreich demonstriert. Durch diverse Befehle kann der Sensor
kalibriert und das Versenden von Messwerten freigeschaltet werden.

\subsection{Fehler}
Die Implementation einer robusten Kommunikation ist komplex. Die ursprüngliche Idee, das Telegrammende ebenfalls durch ein \ac{ASCII}-Zeichen
anzuzeigen, erwies sich als unzuverlässig.Es kam immer wieder zu falsch erkannten Telegrammenden. Ursache dafür war die Nutzung der CRC-Prüfsumme,
da diese durchaus dasselbe Zeichen ergeben kann, welches auch für das Telegrammende definiert war. 

\smallskip

Lösung des Problems war die Übertragung von Längeninformationen und der Verzicht auf ein Ende-Zeichen.

\smallskip

Der STM32 besitzt eine Peripherieeinheit zur Berechnung von CRC32-Prüfsummen. Da der genutzte
Algorithmus keinem etablierten Standard folgt, wurde diese Peripherieeinheit nicht genutzt. Die Implementation der zyklischen Redundanzprüfung mittels 
CRC16-CCITT erwies sich jedoch als zuverlässig.

\subsection{Ausblick}
Diese Arbeit bildet eine Grundlage für zahlreiche mögliche weitere Entwicklungen und Verbesserungen. Es ist durchaus interessant, andere Sensortypen zu implementieren 
und eine einfache Konfiguration dieser zu ermöglichen. Die Verschlüsselung der Kommunikation zwischen den Prozessoren ist eine weitere Möglichkeit, die Datenübertragung
weiter zu optimieren. Schlanke Algorithmen wie XTEA bieten sich hierfür an, da die Berechnung dieser vergleichsweise wenig Rechenleistung benötigt.

\smallskip

Um den Code zu optimieren, wäre die Umstellung auf die Programmiersprache C++ und die damit einhergehende Objektorientierung zukunftsorientiert. Dies würde in Hinblick
auf die Implementation mehrerer Sensoren (auch Sensoren gleichen Typs) den Progammieraufwand verringern und die Anbindung der Sensoren vereinfachen.

\smallskip

Auch der Aspekt der serverseitigen Datenverarbeitung der gesammelten Daten ist interessant. Mittels Projekten wie Node-Red oder anderen Automatisierungsplattformen 
lassen sich \ac{MQTT}-Nachrichten leicht verarbeiten und versenden. Gekoppelt mit einer graphischen Repräsentation der Messwerte ist dies ein logischer 
nächster Schritt in der Fortführung des Projekts.
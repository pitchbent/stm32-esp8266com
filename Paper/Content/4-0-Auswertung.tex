\subsection{Bewertung und Fazit}
Es wurde eine robuste Übertragung sensibler Messwerte erfolgreich implementiert. Mittels der genutzten Technologien wie zyklische Redundanzprüfung, \ac{UART} und \ac{MQTT}
werden die Messwerte zuverlässig übertragen und auf Fehler in der Übertragung mittel zyklischer Redundanzprüfung überprüft. Durch die Nutzung von \ac{MQTT} ist die Anbindung an ein übergeordnetes 
Netzwerk möglich. 

\smallskip

Durch die Implementation eines kapazitiven Feuchtigkeitssensors wurde die Funktionalität der Übertragung erfolgreich demonstriert. Durch diverse Befehle kann der Sensor
kalibriert werden und das Versenden von Messwerten freigeschaltet werden.

\subsection{Fehler}
\paragraph{Nachrichtenformat}
Die Implementation einer gut funktionierenden Kommunikation erwies sich als durchaus komplex. Die ursprüngliche Idee, das Telegrammende ebenfalls durch ein Zeichen
anzuzeigen erwies sich nicht als ausreichend, da es auf Grund des genutzten \ac{ASCII}-Zeichensatzes immer wieder zu falsch erkannten Telegrammenden kam. Dies war 
begründet an der Nutzung der CRC-Prüfsumme, diese kann durchaus das selbe Zeichen ergeben, welches auch für das Ende eines Telegrammes stand. 

\smallskip

Die Übertragung von Längeninformationen und den Verzicht auf ein Stopp-Zeichen löste das Problem.

\paragraph{Zyklische Redundanzprüfung}
Der STM32 besitzt eine Peripherieeinheit zur Berechnung von CRC32-Prüfsummen. Leider erwies sich die Nutzung dieser Peripherieeinheit als zu komplex, da der genutzte
Algorithmus keinen populären Standard folgt und damit die Prüfung der Nachrichten fehlerbehaftet war. Die Implementation der zyklischen Redundanzprüfung mittels 
CRC16-CCITT erwies sich jedoch als zuverlässig.

\subsection{Ausblick}
Diese Arbeit legt einen Grundstein für diverse mögliche weitere Entwicklungen und Verbesserungen. Es ist durchaus interessant, andere Sensortypen zu implementieren 
und eine simple Konfiguration dieser zu ermöglichen. Die Verschlüsselung der Kommunikation zwischen den Prozessoren ist eine weitere Möglichkeit, die Datenübertragung
weiter zu optimieren. Schlanke Algorithmen wie XTEA bieten sich hierfür an, da die Berechnung dieser vergleichsweise wenig Rechenleistung benötigt.

\smallskip

Auch der Aspekt der serverseitigen Datenverarbeitung der gesammelten Daten ist interessant. Mittels Projekten wie Node-Red oder anderen Automatisierungsplattformen 
lassen sich \ac{MQTT}-Nachrichten leicht verarbeiten und zu versenden. Gekoppelt mit einer graphischen Repräsentation der Messwerte ist dies ein logischer 
nächster Schritt in der Fortführung des Projekts.
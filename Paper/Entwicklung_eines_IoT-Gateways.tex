\documentclass[fontsize=11pt,
		 %twoside, 
		 oneside,
		 ngerman,
		 titlepage,
		 paper=a4,
		 bibliography=totoc,
		 listof=totoc,
		 DIV=10,
		 BCOR=5mm,
		 headsepline,
		 footsepline,
		 ]{scrartcl}
		 
%% 8-Bit-Standard Font (Cork-Kodierung)
\usepackage[T1]{fontenc} % 256 Zeichen Fonttabelle, statt 128 Zeichen
%% Direkte Eingabe von Umlauten
\usepackage[utf8]{inputenc}
%% Verwendung der neuen deutschen Rechtschreibung

\usepackage[ngerman]{babel} % babel etwas mehr gewartet, moderner
%
\usepackage{scrhack}
%% Kein Einzug am Absatzbeginn
\setlength{\parindent}{0mm}
%
%% Einbinden von Bildern
\usepackage{setspace}
\usepackage{graphicx}
\usepackage{epstopdf}
\usepackage[absolute]{textpos}
\usepackage{geometry}
\usepackage{multirow}
\usepackage{booktabs} % schöne Tabellenabstände und Linien
\usepackage{textcomp} % griechische/spezielle Buchstaben im Text
%
\usepackage[numbers]{natbib} %mit nummer zitieren
\usepackage[cmyk,table]{xcolor}

\usepackage{acronym}
\usepackage{pdfpages}
\usepackage{color}
\usepackage{listings}

\usepackage{wrapfig}

\usepackage{amsmath} % Zentrierte Formeln

\usepackage{float}   % Bilder besser setzen

\usepackage{url}	 % Längere URL für bibtex
\def\UrlBreaks{\do\/\do-}


%\lstset{numbers=left,numberstyle=\tiny,numbersep=5pt}
\lstset{language = C++}
\lstset{language=C++,
                basicstyle=\ttfamily,
                keywordstyle=\color{blue}\ttfamily,
                stringstyle=\color{red}\ttfamily,
                commentstyle=\color{green}\ttfamily,
                morecomment=[l][\color{magenta}]{\#}
}

%

\usepackage[
pdftitle={Studienarbeit Entwicklung eines mikrocontrollerbasierten IoT-Gateways},%
pdfsubject={Studienarbeit},%
pdfauthor={Julius Bartel},%
%pdfkeywords={{tudienssemester}, {Stichwort 2 der Arbeit}, {Stichwort 3 der Arbeit}},%
%pdfpagelayout=TwoPageRight,%
plainpages=false,%
pdfpagelabels,%
]{hyperref}
\usepackage[nameinlink]{cleveref} % noabbrev % umfangreicher als varioref % intelligente Querverweise
\crefname{subsection}{Kapitel}{Kapitel}
\crefname{subsubsection}{Kapitel}{Kapitel}
%

%Kopf und Fusszeile
\setlength{\headheight}{2cm}
\setlength{\footheight}{1cm}
\usepackage[automark]{scrlayer-scrpage}
%\usepackage{scrpage2}
\clearscrheadfoot
\pagestyle{scrheadings}
\automark[section]{section}
%\ihead[]{\includegraphics[width=0.2\textwidth]{Bilder/hm_4C_L_1-4.eps}}	
%\ohead[]{Entwicklung eines IoT-Gateways\\
%19. Oktober 2020
%}	%[fuer plain configuration]{fuer scrheadings configuration}
%\chead[]{}	%[fuer plain configuration]{fuer scrheadings configuration}
%\ifoot[]{Julius Bartel}	%[fuer plain configuration]{fuer scrheadings configuration}
\cfoot[]{\pagemark}	%[fuer plain configuration]{fuer scrheadings configuration}
%\cfoot[]{}	%[fuer plain configuration]{fuer scrheadings configuration}

\usepackage[hang]{caption}
\usepackage[format=hang,justification=raggedright,singlelinecheck=false]{caption}
\clearcaptionsetup{figure}
\captionsetup[figure]{format=hang,justification=centerlast, singlelinecheck=true}
\usepackage{pgfplots}
\pgfplotsset{
  compat=1.11, %moves axis labels near ticklabels (respects tick label widths)
  grid style={dashed},
  %every linear axis/.append style={
  	%		mark = none, %Seite 161 in der Dokuemntation
 	%		smooth, 
 	%		thick},
 }


\begin{document}
\pagenumbering{arabic}
\begin{titlepage}
\thispagestyle{empty}
%Keine Seitenzahlnummerierung
\begin{center}

%\vspace*{0.5cm}

\begin{figure}[h]
    \centering
        
       \includegraphics[width=0.6\textwidth]{Pictures/hm_4C_L_1-4.eps}
    
 \end{figure}

% \begin{huge}
%    Hochschule Mannheim
%\end{huge}

%\bigskip



\vspace*{2.5cm}
\begin{LARGE}
\textbf{Julius Bartel}
\end{LARGE}

\bigskip

\begin{Huge} 
\textbf{Entwicklung robuster Übertragung sensibler Messwerte}
\end{Huge}


\par\bigskip

\begin{large}
    \textbf{Bachelorarbeit}
\end{large}

\vspace{12cm}
Sommersemester 2021 Betreuer: Prof. Dr.-Ing. Dennis Trebbels




\end{center}

\end{titlepage}

\cleardoublepage
%
\include{Content/0-Selbststaendigkeitserklaerung}
%
\section*{Zusammenfassung}

\addcontentsline{toc}{section}{Zusammenfassung}


Thema dieser Arbeit ist die Entwicklung einer robusten Kommunikation zwischen zwei \acp{uC} zur Weiterleitung von sensiblen Messdaten und 
der anschließenden Weiterleitung der Daten an ein übergeordnetes Netzwerk. Die Entwicklung soll an Hand eines Praxisbeispiels demonstriert werden.

\smallskip

Grundlage der Arbeit ist meine vorhergegangene Studienarbeit \citep{IoTGateway}, in welcher die Hardware,
welche die Grundlage für dieses Projekt bildet, entwickelt, bestückt und in Betrieb genommen wurde.

Die entwickelte Platine verfügt über zwei \acs{uC}, deren \acs{UART}-Schnittstellen miteinander verbunden sind. Ein Prozessor ist hierbei
für die Verarbeitung von Sensormesswerten zuständig, während der andere als \acs{WLAN}-Relais an ein übergeordnetes Netzwerk fungiert.

\smallskip

Auf Basis der \acs{UART}-Schnittstellen der beiden Prozessoren soll ein Protokoll implementiert werden, welches sowohl den Transport
von Messwerten in die eine Richtung, als auch den Transport von Konfigurationsbefehlen in die andere Richtung ermöglicht. 

Als Praxisbeispiel wurde die kapazitive Messung von Feuchtigkeit im Erdreich gewählt. Die gemessenen Daten sollen verarbeitet und
per \ac{MQTT} an ein Netzwerk weitergeleitet werden.

\smallskip

Der gesamte Code ist in einer GitHub-Repository zu finden \citep{Listing}.
\section*{Abkuerzungsverzeichnis}

\begin{acronym}
    \acro{HAL}{Hardware Abstraction Layer}
    \acro{IoT}{Internet of Things \textit{en. Internet der Dinge}}
    \acro{uC}[\textmu C]{Microcontroller}
    \acro{WLAN}{Wireless Local Area Network \textit{en. Drahtlose Netzwerkverbindung}}
    \acro{RAM}{Random Access Memory}
    \acro{ADC}{Analog Digital Convert \textit{en. Analog-Digital Wandler}}
    \acro{JTAG}{Joint Test Action Group}
    \acro{UART}{Universal Asynchronous Receiver Transmitter}
    \acro{USART}{Uinversal Synchronous Receiver Transmitter}
    \acro{USB}{Universal Serial Bus}
    \acro{MQTT}{Message Queuing Telemetry Transport}
    \acro{SPI}{Serial Periphal Interface}
    \acro{I2C}{Inter-Integrated Circuit}
    \acro{DMA}{Direct Memory Access}
    \acro{AHB-Bus}{Advanced High-Perfomance Bus}
    \acro{GPIO}{General Purpose Input Output}
    \acro{MSB}{Most Significant Bit}
    \acro{LSB}{Least Significant Bit}
    \acro{FIFO}{First In First Out}
    \acro{ASCII}{American Standard Code for Information Interchange}

\end{acronym}

\tableofcontents
\clearpage

\section{Aufgabenstellung}
\label{sec: Aufgabenstellung}
\input{Content/1-1-Erklaerung}
\input{Content/1-2-Anforderung}
\clearpage

\section{Hintergrund und Ausgangslage}
\label{sec:Hintergrund und Ausgangslage}
\input{Content/2-1-esp8266}
\input{Content/2-2-stm32f103}
\input{Content/2-3-Altium_Designer}
\input{Content/2-4-Linearregler}
\clearpage

\section{Umsetzung}
\label{sec:Umsetzung}
\input{Content/3-1-Schaltplaene}
\input{Content/3-2-Layout}
\clearpage


\section{Bestückung und Inbetriebnahme}
\label{sec:BestueckungInbetriebnahme}
\input{Content/4-1-BestueckungInbetrieb}
\clearpage

\section{Programmierung}
\label{sec:Programmierung}
\input{Content/5-1-Programmierung}
\clearpage

\section{Fazit}
\label{fazit}
\input{Content/6-1-Fazit}
\clearpage
% \input{Inhalt/2-1-Ventilsteuerung-Aufgabenstellung}

% \input{Inhalt/2-1-Ventilsteuerung-Umsetzung}

% \input{Inhalt/2-1-Ventilsteuerung-Auslegung}
% %\input{Inhalt/2-2-Projekt1-FazitAusblick}
% \clearpage
% \section{Projekt 2: Arduino OWP-Programmer}
% \label{sec:OWP}
% \input{Inhalt/3-1-Programmer-Aufgabenstellung}
% \clearpage
%\input{Inhalt/3-1-Programmer-Umsetzung}
%\clearpage
%\input{Inhalt/3-1-Programmer-Programmieradapter}

%\input{Inhalt/4-1-Programmierung}
%\clearpage
%\include{Inhalt/4-Fazit}


%include list of figure
\listoffigures
\clearpage
%
%include list of tables
%\listoftables
%\clearpage

%Bibliotheksüberschrift anpassen
\bibliographystyle{unsrtnat} % plain, alpha, dinat
%Ändern der Überschrift des Literaturverzeichnisses zu Quellenverzeichnis
\renewcommand{\refname}{Quellenverzeichnis}
\bibliography{bibliography}


\clearpage

%\section{Anhang}
%\subsection{Schaltpläne}
\clearpage
\addcontentsline{toc}{section}{Anhang} 

\include{Anhang/1-Anhang}


\mbox{}
\end{document}
